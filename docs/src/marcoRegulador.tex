\chapter{Marco Regulador \label{sec:MarcoRegulador}}

\section{Legislación aplicable}
Como se ha comentado anteriormente, la tecnología \textit{big data} permite el procesado de grandes cantidades de datos de forma más rápida que con tecnologías anteriores, con el propósito de obtener información valiosa de ella y generar conocimiento.

Este gran volumen de datos disponibles proviene de diferentes fuentes de información, que abarcan desde reportes anónimos hasta informes de uso de usuarios, donde estos están perfectamente identificados. Son este último tipo de datos los que pueden producir problemas legales, debido a la legislación vigente respecto a la privacidad de las personas.

La normativa nacional establece que ``un dato de carácter personal es cualquier información que permita identificarte o hacerte identificable'' y, por ello, ``reconoce al ciudadano la facultad de controlar sus datos personales y la capacidad para disponer y decidir sobre los mismos'' mediante el derecho fundamental a la protección de datos \cite{aepd}.

En España, es la \gls{AEPD} ``la autoridad de control independiente que vela por el cumplimiento de la normativa sobre protección de datos''. Además, ``garantiza y tutela el derecho fundamental a la protección de datos personales'' \cite{aepd}.

En la actualidad, es la Ley Orgánica 15/1999, de 13 de diciembre, de Protección de Datos de Carácter Personal \cite{leyPrivacidad} la que afecta al procesamiento de información que se realiza en los sistemas \textit{big data}. Esta, tiene que objetivo ``garantizar
y  proteger,  en  lo  que  concierne  al  tratamiento  de  los
datos personales, las libertades públicas y los derechos
fundamentales de las personas físicas, y especialmente
de su honor e intimidad personal y familiar'' \cite{leyPrivacidad}. Los derechos que incluye esta ley son \cite{derechosCiu}:

\clearpage
\begin{itemize}
\item Derecho de información: En el momento en que se procede a la recogida de los datos personales, el interesado debe ser informado previamente. 

\item Derecho de acceso: permite al ciudadano conocer y obtener gratuitamente información sobre sus datos de carácter personal que han sido tratados.

\item Derecho de rectificación: permite corregir errores, modificar los datos que resulten ser inexactos o incompletos y garantizar la certeza de la información tratada.

\item Derecho de cancelación: permite que se supriman los datos que resulten ser inadecuados o excesivos.

\item Derecho de oposición: permite al afectado que el tratamiento de sus datos de carácter personal no se realice o el cese de los mismo.
\end{itemize}

Con respecto a la Unión Europea, la ley que actualmente regula la protección y privacidad de los datos de carácter personal es el Reglamento (UE) 2016/679 del Parlamento Europeo y del Consejo de 27 de abril de 2016, relativo a la protección de las personas físicas en lo que respecta al tratamiento de datos personales y a la libre circulación de estos datos \cite{lawEU}.

Esta legislación europea ha sido reformada recientemente y se espera que para 2018 se aplique totalmente en todos los países de la zona euro. En esta reforma destacan, entre otras, la reforma al \Gls{derOlvido}, las grandes multas por el incumplimiento de las leyes de privacidad y el endurecimiento de los controles parentales \cite{rulesEU}. En general, esta reforma ha sido un endurecimiento de las medidas ya existentes para salvaguardar los derechos de los ciudadanos en esta nueva época digital.

Por tanto, con respecto al marco regulador que afecta a las aplicaciones del \textit{big data} se han de tener en cuenta aspectos como la seguridad, privacidad y correcta conservación de los datos personales de los usuarios, para proteger los derechos de los ciudadanos. De la misma forma, también debe asegurarse la transparencia de su uso frente a las autoridades.

\clearpage
\section{Estándares técnicos}
La utilización del \textit{Big Data} de forma masiva para el análisis de datos en el mundo empresarial, como ya se ha comentado, es una técnica relativamente reciente, por lo que no se ha producido un gran desarrollo con respecto a estándares de uso.

El primer estándar que se desarrolló data de noviembre de 2015, cuando la \gls{ITU}, agencia que depende de la \gls{ONU} y que es responsable de los problemas que conciernen a las tecnologías de información y comunicación, desarrolló y publicó el documento ITU-T Y.3600 (11/2015) \cite{estandar} titulado ``Big data - Cloud computing based requirements and capabilities''.

En este documento ``se presentan los requisitos, las capacidades y los casos de uso de los volúmenes masivos de datos (\textit{big data}) de computación en la nube, así como su contexto de sistema'' \cite{estandar}. Además, en este documento se establece la definición de \textit{big data} comentada en el apartado \ref{defBigData}.

